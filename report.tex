% use "xelatex -shell-escape report.tex" to compile
\documentclass[titlepage,12pt]{article}
\usepackage{fontspec}
\usepackage{minted}
\setminted{linenos,fontsize=\scriptsize}
\usepackage{amsmath}
\usepackage{amssymb}

\usepackage[margin=25mm]{geometry}

\title{\textbf{Dynamic Programming Project} \\ {\small Report for \emph{CSCI-406:
Algorithms} with Dr.~Dinesh Mehta}}
\author{%
    Peter Palumbo \\ {\small\texttt{ppalumbo@mines.edu}} \\[12pt]
    Jack Rosenthal \\ {\small\texttt{jrosenth@mines.edu}} \\[12pt]
    Paul Sattizahn \\ {\small\texttt{psattiza@mines.edu}} \\
}
\date{4 November 2016}

\setlength\parindent{0pt}
\setlength\parskip{6pt plus 2pt minus 2pt}
\let\ge=\geqslant
\let\le=\leqslant

\begin{document}

\maketitle
\section{Introduction}

%TODO: brief (one paragraph) introduction to the problem

\section{Recursive Formulation}

\begin{itemize}
    \item Let $\mathtt{evset}_p\{t\}$ denote the set containing only $t$ if an event
        at time $t$ occurs at coordinate $p$, otherwise the empty set.
    \item Let $n$ denote the number of events to occur.
    \item Let $c$ denote the coordinate of the last event.
    \item Let $P(t, x)$ be the predicate $n - t \ge |x - c|$.
\end{itemize}

Our recursive formulation for the maximum observable events subset, $m(t, p)$
where $t$ is the current time and $p$ is the current coordinate, is as
follows.

\begin{displaymath}
    m(t, p) = \mathtt{evset}_p\{t\}\, \cup\,
    \begin{cases}
        \emptyset & \text{if}\ t = n \\
        \text{max}_{\text{length}} \left\{m(t + 1, x) :
        x \in \left\{p - 1, p, p + 1\right\} | P(t, x)
        \right\} & \text{otherwise}
    \end{cases}
\end{displaymath}

Using the above recursive formula, the maximum observable events subset for the
problem presented is $m(0, 0)$.

\subsection{Explanation}

%TODO

\subsection{Recurrence Relation}

%TODO

\subsection{Psuedocode}

\inputminted{python3}{dp.py}

\subsection{Algorithmic Complexity}

%TODO: explain why it's O(n^2)

\section{Implementation}

Our implementation in the \emph{Python} programming language is as follows:

\inputminted{python3}{dp.py}

\subsection{Examples}

%TODO: examples

\end{document}
